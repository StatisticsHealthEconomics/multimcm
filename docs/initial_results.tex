% Options for packages loaded elsewhere
\PassOptionsToPackage{unicode}{hyperref}
\PassOptionsToPackage{hyphens}{url}
%
\documentclass[
]{article}
\usepackage{lmodern}
\usepackage{amssymb,amsmath}
\usepackage{ifxetex,ifluatex}
\ifnum 0\ifxetex 1\fi\ifluatex 1\fi=0 % if pdftex
  \usepackage[T1]{fontenc}
  \usepackage[utf8]{inputenc}
  \usepackage{textcomp} % provide euro and other symbols
\else % if luatex or xetex
  \usepackage{unicode-math}
  \defaultfontfeatures{Scale=MatchLowercase}
  \defaultfontfeatures[\rmfamily]{Ligatures=TeX,Scale=1}
\fi
% Use upquote if available, for straight quotes in verbatim environments
\IfFileExists{upquote.sty}{\usepackage{upquote}}{}
\IfFileExists{microtype.sty}{% use microtype if available
  \usepackage[]{microtype}
  \UseMicrotypeSet[protrusion]{basicmath} % disable protrusion for tt fonts
}{}
\makeatletter
\@ifundefined{KOMAClassName}{% if non-KOMA class
  \IfFileExists{parskip.sty}{%
    \usepackage{parskip}
  }{% else
    \setlength{\parindent}{0pt}
    \setlength{\parskip}{6pt plus 2pt minus 1pt}}
}{% if KOMA class
  \KOMAoptions{parskip=half}}
\makeatother
\usepackage{xcolor}
\IfFileExists{xurl.sty}{\usepackage{xurl}}{} % add URL line breaks if available
\IfFileExists{bookmark.sty}{\usepackage{bookmark}}{\usepackage{hyperref}}
\hypersetup{
  pdftitle={Bayesian Mixture Cure Modelling in Metastatic Melanoma (CheckMate 067)},
  pdfauthor={N Green},
  hidelinks,
  pdfcreator={LaTeX via pandoc}}
\urlstyle{same} % disable monospaced font for URLs
\usepackage[margin=1in]{geometry}
\usepackage{graphicx}
\makeatletter
\def\maxwidth{\ifdim\Gin@nat@width>\linewidth\linewidth\else\Gin@nat@width\fi}
\def\maxheight{\ifdim\Gin@nat@height>\textheight\textheight\else\Gin@nat@height\fi}
\makeatother
% Scale images if necessary, so that they will not overflow the page
% margins by default, and it is still possible to overwrite the defaults
% using explicit options in \includegraphics[width, height, ...]{}
\setkeys{Gin}{width=\maxwidth,height=\maxheight,keepaspectratio}
% Set default figure placement to htbp
\makeatletter
\def\fps@figure{htbp}
\makeatother
\setlength{\emergencystretch}{3em} % prevent overfull lines
\providecommand{\tightlist}{%
  \setlength{\itemsep}{0pt}\setlength{\parskip}{0pt}}
\setcounter{secnumdepth}{-\maxdimen} % remove section numbering

\title{Bayesian Mixture Cure Modelling in Metastatic Melanoma (CheckMate
067)}
\author{N Green}
\date{14/10/2020}

\begin{document}
\maketitle

{[}taken from Anthonio{]} \#\#\# Methodology Immuno-oncologic (IO)
studies for melanoma therapies, such as \emph{ipilimumab},
\emph{nivolumab}, and the \emph{nivolumab} + \emph{ipilimumab}
combination, have indicated that survival curves ``plateau'' (a
considerable proportion of patients are ``long-term survivors''). Cure
models are a special type of survival analysis where this ``cure
fraction'' (the underlying proportion of responders to
treatment/long-term survivors) is accounted for. Cure models estimate
the cure fraction, in addition to a parametric survival function for
patients that are not cured. The mortality risk in the cured patients is
informed by a background mortality rate. The population that is not
cured is subject both to background mortality and to additional
mortality from their cancer, estimated using a parametric survival
model.

A mixture cure model (MCM) is a type of cure model where survival is
modelled as a mixture of two groups of patients: those who are cured and
those who are not (and who therefore remain at risk). The survival for a
population with a cure fraction can be written as follows:

\[
S(t, x) = S^*(t, x)[\pi(x) + (1 − \pi(x))S_u(t, x)],
\]

where \(S(t, x)\) denotes the survival at time \(t\), \(S^*(t, x)\)
denotes the background mortality at time \(t\) conditional on covariates
\(x\), \(\pi(x)\) denotes the probability of being cured conditional on
covariates \(x\), and \(S_u(t, x)\) denotes the mortality due to cancer
at time t conditional on covariates \(x\). We use World Health
Organization (WHO) life tables by country (2018) to inform the
background mortality rate (baseline hazard) inputed to the cure model.
Such baseline hazard is the expected mortality rate for each patient at
the age at which he/she experiences the event. The mortality data are
age- and gender adjusted, thus providing a granular account of the
different patient profiles in the trial. Note that WHO reports
conditional probabilities of death in 5-year intervals until age 85. A
constant annual mortality rate is reported for individuals over 85. In
addition, we make the assumption that the maximum age that a patient can
live up to is 100 years.

To model the disease-specific mortality (the uncured fraction), all
standard parametric distributions are tested:

\begin{itemize}
\tightlist
\item
  exponential
\item
  Weibull
\item
  Gompertz
\item
  log-normal
\item
  log-logistic
\item
  generalised gamma.
\end{itemize}

Parameters for the models, including the cure rate parameter, are
derived via Bayesian inference.

\end{document}
