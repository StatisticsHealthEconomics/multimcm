% Options for packages loaded elsewhere
\PassOptionsToPackage{unicode}{hyperref}
\PassOptionsToPackage{hyphens}{url}
%
\documentclass[
]{article}
\usepackage{lmodern}
\usepackage{amsmath}
\usepackage{ifxetex,ifluatex}
\ifnum 0\ifxetex 1\fi\ifluatex 1\fi=0 % if pdftex
  \usepackage[T1]{fontenc}
  \usepackage[utf8]{inputenc}
  \usepackage{textcomp} % provide euro and other symbols
  \usepackage{amssymb}
\else % if luatex or xetex
  \usepackage{unicode-math}
  \defaultfontfeatures{Scale=MatchLowercase}
  \defaultfontfeatures[\rmfamily]{Ligatures=TeX,Scale=1}
\fi
% Use upquote if available, for straight quotes in verbatim environments
\IfFileExists{upquote.sty}{\usepackage{upquote}}{}
\IfFileExists{microtype.sty}{% use microtype if available
  \usepackage[]{microtype}
  \UseMicrotypeSet[protrusion]{basicmath} % disable protrusion for tt fonts
}{}
\makeatletter
\@ifundefined{KOMAClassName}{% if non-KOMA class
  \IfFileExists{parskip.sty}{%
    \usepackage{parskip}
  }{% else
    \setlength{\parindent}{0pt}
    \setlength{\parskip}{6pt plus 2pt minus 1pt}}
}{% if KOMA class
  \KOMAoptions{parskip=half}}
\makeatother
\usepackage{xcolor}
\IfFileExists{xurl.sty}{\usepackage{xurl}}{} % add URL line breaks if available
\IfFileExists{bookmark.sty}{\usepackage{bookmark}}{\usepackage{hyperref}}
\hypersetup{
  pdftitle={How to use},
  hidelinks,
  pdfcreator={LaTeX via pandoc}}
\urlstyle{same} % disable monospaced font for URLs
\usepackage[margin=1in]{geometry}
\usepackage{color}
\usepackage{fancyvrb}
\newcommand{\VerbBar}{|}
\newcommand{\VERB}{\Verb[commandchars=\\\{\}]}
\DefineVerbatimEnvironment{Highlighting}{Verbatim}{commandchars=\\\{\}}
% Add ',fontsize=\small' for more characters per line
\usepackage{framed}
\definecolor{shadecolor}{RGB}{248,248,248}
\newenvironment{Shaded}{\begin{snugshade}}{\end{snugshade}}
\newcommand{\AlertTok}[1]{\textcolor[rgb]{0.94,0.16,0.16}{#1}}
\newcommand{\AnnotationTok}[1]{\textcolor[rgb]{0.56,0.35,0.01}{\textbf{\textit{#1}}}}
\newcommand{\AttributeTok}[1]{\textcolor[rgb]{0.77,0.63,0.00}{#1}}
\newcommand{\BaseNTok}[1]{\textcolor[rgb]{0.00,0.00,0.81}{#1}}
\newcommand{\BuiltInTok}[1]{#1}
\newcommand{\CharTok}[1]{\textcolor[rgb]{0.31,0.60,0.02}{#1}}
\newcommand{\CommentTok}[1]{\textcolor[rgb]{0.56,0.35,0.01}{\textit{#1}}}
\newcommand{\CommentVarTok}[1]{\textcolor[rgb]{0.56,0.35,0.01}{\textbf{\textit{#1}}}}
\newcommand{\ConstantTok}[1]{\textcolor[rgb]{0.00,0.00,0.00}{#1}}
\newcommand{\ControlFlowTok}[1]{\textcolor[rgb]{0.13,0.29,0.53}{\textbf{#1}}}
\newcommand{\DataTypeTok}[1]{\textcolor[rgb]{0.13,0.29,0.53}{#1}}
\newcommand{\DecValTok}[1]{\textcolor[rgb]{0.00,0.00,0.81}{#1}}
\newcommand{\DocumentationTok}[1]{\textcolor[rgb]{0.56,0.35,0.01}{\textbf{\textit{#1}}}}
\newcommand{\ErrorTok}[1]{\textcolor[rgb]{0.64,0.00,0.00}{\textbf{#1}}}
\newcommand{\ExtensionTok}[1]{#1}
\newcommand{\FloatTok}[1]{\textcolor[rgb]{0.00,0.00,0.81}{#1}}
\newcommand{\FunctionTok}[1]{\textcolor[rgb]{0.00,0.00,0.00}{#1}}
\newcommand{\ImportTok}[1]{#1}
\newcommand{\InformationTok}[1]{\textcolor[rgb]{0.56,0.35,0.01}{\textbf{\textit{#1}}}}
\newcommand{\KeywordTok}[1]{\textcolor[rgb]{0.13,0.29,0.53}{\textbf{#1}}}
\newcommand{\NormalTok}[1]{#1}
\newcommand{\OperatorTok}[1]{\textcolor[rgb]{0.81,0.36,0.00}{\textbf{#1}}}
\newcommand{\OtherTok}[1]{\textcolor[rgb]{0.56,0.35,0.01}{#1}}
\newcommand{\PreprocessorTok}[1]{\textcolor[rgb]{0.56,0.35,0.01}{\textit{#1}}}
\newcommand{\RegionMarkerTok}[1]{#1}
\newcommand{\SpecialCharTok}[1]{\textcolor[rgb]{0.00,0.00,0.00}{#1}}
\newcommand{\SpecialStringTok}[1]{\textcolor[rgb]{0.31,0.60,0.02}{#1}}
\newcommand{\StringTok}[1]{\textcolor[rgb]{0.31,0.60,0.02}{#1}}
\newcommand{\VariableTok}[1]{\textcolor[rgb]{0.00,0.00,0.00}{#1}}
\newcommand{\VerbatimStringTok}[1]{\textcolor[rgb]{0.31,0.60,0.02}{#1}}
\newcommand{\WarningTok}[1]{\textcolor[rgb]{0.56,0.35,0.01}{\textbf{\textit{#1}}}}
\usepackage{graphicx}
\makeatletter
\def\maxwidth{\ifdim\Gin@nat@width>\linewidth\linewidth\else\Gin@nat@width\fi}
\def\maxheight{\ifdim\Gin@nat@height>\textheight\textheight\else\Gin@nat@height\fi}
\makeatother
% Scale images if necessary, so that they will not overflow the page
% margins by default, and it is still possible to overwrite the defaults
% using explicit options in \includegraphics[width, height, ...]{}
\setkeys{Gin}{width=\maxwidth,height=\maxheight,keepaspectratio}
% Set default figure placement to htbp
\makeatletter
\def\fps@figure{htbp}
\makeatother
\setlength{\emergencystretch}{3em} % prevent overfull lines
\providecommand{\tightlist}{%
  \setlength{\itemsep}{0pt}\setlength{\parskip}{0pt}}
\setcounter{secnumdepth}{-\maxdimen} % remove section numbering
\ifluatex
  \usepackage{selnolig}  % disable illegal ligatures
\fi

\title{How to use}
\author{}
\date{\vspace{-2.5em}}

\begin{document}
\maketitle

\hypertarget{introduction}{%
\subsection{Introduction}\label{introduction}}

This is a basic introduction to how to use \texttt{rstanbmcm} to fit
Bayesian mixture cure models in Stan.

\hypertarget{data}{%
\subsection{Data}\label{data}}

We will use the Checkmate 067 study data set. The data have already been
arranged in to the correct format and saved within the package so we can
load it as follows.

\begin{Shaded}
\begin{Highlighting}[]
\FunctionTok{data}\NormalTok{(}\StringTok{"surv\_input\_data"}\NormalTok{, }\AttributeTok{package =} \StringTok{"rstanbmcm"}\NormalTok{)}
\end{Highlighting}
\end{Shaded}

There should be event times and censoring indicators for both OS and
PFS. There should be a treatment label column. Additional patient-level
coviarate can also be included. At present only age at event is used.

This looks like this.

\begin{Shaded}
\begin{Highlighting}[]
\FunctionTok{head}\NormalTok{(surv\_input\_data)}
\CommentTok{\#\textgreater{}   OSage PFSage        os os\_event       pfs pfs\_event                 TRTA SEX COUNTRY    ACOUNTRY}
\CommentTok{\#\textgreater{} 1    57     56 60.024641        0 59.663244         0 NIVOLUMAB+IPILIMUMAB   M     NLD NETHERLANDS}
\CommentTok{\#\textgreater{} 2    78     77 19.449692        1  2.628337         1            NIVOLUMAB   F     NLD NETHERLANDS}
\CommentTok{\#\textgreater{} 3    67     67  2.069815        1  2.069815         1           IPILIMUMAB   M     NLD NETHERLANDS}
\CommentTok{\#\textgreater{} 4    48     47 60.188912        0 59.958932         0            NIVOLUMAB   F     NLD NETHERLANDS}
\CommentTok{\#\textgreater{} 5    76     73 64.262834        0 32.295688         1            NIVOLUMAB   M     NLD NETHERLANDS}
\CommentTok{\#\textgreater{} 6    78     76 34.891170        1  2.562628         1           IPILIMUMAB   M     NLD NETHERLANDS}
\CommentTok{\#\textgreater{}   PFS\_rate OS\_rate}
\CommentTok{\#\textgreater{} 1    0.005   0.005}
\CommentTok{\#\textgreater{} 2    0.026   0.026}
\CommentTok{\#\textgreater{} 3    0.014   0.014}
\CommentTok{\#\textgreater{} 4    0.001   0.001}
\CommentTok{\#\textgreater{} 5    0.023   0.041}
\CommentTok{\#\textgreater{} 6    0.041   0.041}
\end{Highlighting}
\end{Shaded}

\hypertarget{example}{%
\subsection{Example}\label{example}}

First of all attach all of the libraries we are going to need.

\begin{Shaded}
\begin{Highlighting}[]
\FunctionTok{library}\NormalTok{(purrr)}
\FunctionTok{library}\NormalTok{(reshape2)}
\FunctionTok{library}\NormalTok{(dplyr)}
\FunctionTok{library}\NormalTok{(rstan)}
\FunctionTok{library}\NormalTok{(shinystan)}
\FunctionTok{library}\NormalTok{(dplyr)}
\FunctionTok{library}\NormalTok{(ggplot2)}
\FunctionTok{library}\NormalTok{(rstanbmcm)}
\end{Highlighting}
\end{Shaded}

For demonstration purposes we will select a single treatment and fit
Exponential distributions to both OS and PFS.

\begin{Shaded}
\begin{Highlighting}[]
\NormalTok{i }\OtherTok{\textless{}{-}}  \StringTok{"exp"}
\NormalTok{k }\OtherTok{\textless{}{-}} \StringTok{"exp"}
\NormalTok{j }\OtherTok{\textless{}{-}} \StringTok{"IPILIMUMAB"}
\end{Highlighting}
\end{Shaded}

To use the Stan engine we set some options to use all but one of the
available cores and not to over-write pre-complied code.

\begin{Shaded}
\begin{Highlighting}[]
\FunctionTok{rstan\_options}\NormalTok{(}\AttributeTok{auto\_write =} \ConstantTok{TRUE}\NormalTok{)}
\FunctionTok{options}\NormalTok{(}\AttributeTok{mc.cores =}\NormalTok{ parallel}\SpecialCharTok{::}\FunctionTok{detectCores}\NormalTok{() }\SpecialCharTok{{-}} \DecValTok{1}\NormalTok{)}
\end{Highlighting}
\end{Shaded}

Now we are ready to do the model fitting. There are 2 options to use.

\begin{itemize}
\tightlist
\item
  \texttt{bmcm\_joint\_stan\_file}: calls the Stan file directly from R
  without pre-compiling. This is useful for development.
\item
  \texttt{bmcm\_joint\_stan}: uses the pre-compiled Stan code.
\end{itemize}

An example call to \texttt{bmcm\_joint\_stan\_file} is given below.

\begin{Shaded}
\begin{Highlighting}[]
\NormalTok{out }\OtherTok{\textless{}{-}}
  \FunctionTok{bmcm\_joint\_stan\_file}\NormalTok{(}
    \AttributeTok{input\_data =}\NormalTok{ surv\_input\_data,}
    \AttributeTok{model\_os =}\NormalTok{ i,}
    \AttributeTok{model\_pfs =}\NormalTok{ k,}
    \AttributeTok{tx\_name =}\NormalTok{ j,}
    \AttributeTok{params\_pfs =} \FunctionTok{list}\NormalTok{(}\AttributeTok{mu\_0 =} \FunctionTok{c}\NormalTok{(}\SpecialCharTok{{-}}\DecValTok{3}\NormalTok{, }\DecValTok{0}\NormalTok{),}
                      \AttributeTok{sigma\_0 =} \FunctionTok{c}\NormalTok{(}\FloatTok{0.5}\NormalTok{, }\FloatTok{0.01}\NormalTok{)),}
    \AttributeTok{params\_os =} \FunctionTok{list}\NormalTok{(}\AttributeTok{mu\_0 =} \FunctionTok{c}\NormalTok{(}\SpecialCharTok{{-}}\DecValTok{3}\NormalTok{, }\DecValTok{0}\NormalTok{),}
                     \AttributeTok{sigma\_0 =} \FunctionTok{c}\NormalTok{(}\FloatTok{0.4}\NormalTok{, }\DecValTok{1}\NormalTok{)),}
    \AttributeTok{params\_cf =} \FunctionTok{list}\NormalTok{(}\AttributeTok{mu\_cf\_os =} \FunctionTok{array}\NormalTok{(}\SpecialCharTok{{-}}\FloatTok{0.8}\NormalTok{, }\DecValTok{1}\NormalTok{),}
                     \AttributeTok{mu\_cf\_pfs =} \FunctionTok{array}\NormalTok{(}\SpecialCharTok{{-}}\FloatTok{0.8}\NormalTok{, }\DecValTok{1}\NormalTok{),}
                     \AttributeTok{sd\_cf\_os =} \FunctionTok{array}\NormalTok{(}\FloatTok{0.5}\NormalTok{, }\DecValTok{1}\NormalTok{),}
                     \AttributeTok{sd\_cf\_pfs =} \FunctionTok{array}\NormalTok{(}\FloatTok{0.5}\NormalTok{, }\DecValTok{1}\NormalTok{)),}
    \AttributeTok{cf\_model =} \DecValTok{2}\NormalTok{,}
    \AttributeTok{joint\_model =} \ConstantTok{FALSE}\NormalTok{,}
    \AttributeTok{warmup =} \DecValTok{100}\NormalTok{,}
    \AttributeTok{iter =} \DecValTok{1000}\NormalTok{,}
    \AttributeTok{thin =} \DecValTok{10}\NormalTok{)}
\end{Highlighting}
\end{Shaded}

\hypertarget{explanation-of-function-arguments}{%
\subsubsection{Explanation of function
arguments}\label{explanation-of-function-arguments}}

\begin{itemize}
\item
  The first thing to note is that we supply the study data as the first
  argument \texttt{inpout\_data}. We then define the which distributions
  we want to fit to the OS anf PFS data, followed by the particular
  treatment subset of data to use from \texttt{input\_data}.
\item
  The next 3 arguments \texttt{params\_pfs}, \texttt{params\_os} and
  \texttt{params\_cf} are the prior parameters for the PFS, OS and cure
  fraction distributions respectively. These must be supplied as a list.
  The two values for each parameter corresponds to the intercept and age
  effect in the linear equation component of the rate regression. For
  the parameters that are optional we have to wrap them with
  \texttt{array(.,1)} because stan expects an array object even when it
  is of dimension (1,1). The cure fraction parameters here are optional
  because there are alternative ways of defining its prior i.e.~using a
  Beta distribution or using the same prior for both OS and PFS. This is
  the separate cure fraction model.
\item
  \texttt{cf\_model} defines whether this is a pooled (1), separate (2)
  or hierarchical (3) cure fraction model.
\item
  \texttt{joint\_model} is a logical argument defining whether we model
  the OS and PFS event times jointly. If \texttt{joint\_model\ =\ TRUE}
  then we must also pass the prior parameters using the
  \texttt{params\_joint} argument.
\item
  Finally, the remaining arguments are passed directly to the Stan
  engine.
\end{itemize}

\end{document}
